\documentclass{TDP003mall}
\usepackage{graphicx}


\newcommand{\version}{Version 1.0}
\author{Jonathan Dandemar, \url{jonda429@student.liu.se}\\
 }
\title{Refelektionsdokument}
\date{2015-12-19}
\rhead{Jonathan Dandemar\\
}

\begin{document}
\projectpage
\section*{Revisionshistorik}
\begin{table}[!h]
\begin{tabularx}{\linewidth}{|l|X|l|}
\hline
Ver. & Revisionsbeskrivning & Datum \\\hline
1.0 & Skapade dokumentet & 2015-12-19 \\\hline
\end{tabularx}
\end{table}

\newpage

\section{Ny programmeringskunskap}
Projektet har varit väldigt lärorik för min utveckling som programmerare då det har gjort att jag har fått mer kunskap och vana 
med att programmera i C++. Det har varit givande att använda SDL för att få en bättre förståelse för hur grafik
och programmering kan vara sammankopplat, jag vet inte hur andra liknande ramverk fungerar men SDL verkar som en bra grund
att arbeta med på den här nivån. Det känns även kul att veta att jag nu kan utveckla den här typen av program som faktiskt "syns"
mer än som text i ett terminalfönster och som. 

Spelet som vi har skapat innehåller väldigt mycket rotation och vinklar då olika objekt flyger åt olika håll och snurrar och byter 
riktning. Det har inneburit att jag har behövt läsa upp på gamla trigonometrikunskaper vilket har varit förvånansvärt intressant
och framförallt så känns det som att det kan vara bra kunskap att ha nära till hands för liknande projekt i framtiden eftersom
matten är densamma även vid byte av programmeringsspråk.  

Många koncept som inte var helt klara innan har blivit lite tydligare nu även om jag känner att jag
fortfarande skulle behöva mer erfarenhet för att känna mig helt bekväm med att jobba i C++. Framförallt så skulle 


\end{document}
