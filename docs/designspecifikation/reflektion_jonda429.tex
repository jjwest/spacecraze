\documentclass{TDP003mall}
\usepackage{graphicx}


\newcommand{\version}{Version 1.0}
\author{Jonathan Dandemar, \url{jonda429@student.liu.se}\\
 }
\title{Refelektionsdokument}
\date{2015-12-19}
\rhead{Jonathan Dandemar\\
}

\begin{document}
\projectpage
\section*{Revisionshistorik}
\begin{table}[!h]
\begin{tabularx}{\linewidth}{|l|X|l|}
\hline
Ver. & Revisionsbeskrivning & Datum \\\hline
1.0 & Skapade dokumentet & 2015-12-19 \\\hline
\end{tabularx}
\end{table}

\newpage

\section{Arbetsgång}
Vi skulle egentligen vara tre personer i gruppen men ena medlemmen var frånvarande under i stort sett hela projektets gång.
Det ledde till att början av projektet blev ganska jobbigt eftersom det var svårt att veta hur vi skulle dela upp arbetet
när ena medlemmen var med ena dagen och borta nästa. Vi insåg dock ganska tidigt att det var enklast att räkna med att vi 
skulle göra projektet på två personer och efter det så fortgick arbetet bättre. Det var dock ett tråkigt problem att behöva ha 
att göra med och jag tror tyvärr inte att jag är så mycket klokare av hur liknande situationer ska hanteras. Det känns som att
jag möjligtvis hade kunnat hjälpa till mer med att få in medlemmen i gruppen men samtidigt så fick han en uppgift att göra precis 
som oss utan att göra den vilket ledde till att jag fick göra den istället och sådana situationer är av naturliga skäl ganska
frustrerande. Självklart vill man förstå och hjälpa någon som inte är helt med men samtidigt så tycker jag att alla borde kunna 
ta sitt egna ansvar för att bidra till ett projekt på den här nivån, eller iallafall försöka, vilket inte var fallet den här gången. \newline \newline 
Vi började projektet med att göra en OOA för att få en överblick av klasserna vi trodde att vi skulle behöva i spelet, klasserna 
skrev vi upp på post-it lappar som vi använde för att visualisera hierarkin vilket jag tyckte var enormt givande då det gjorde
så många andra saker lättare jämfört med att börja koda direkt. Nu behövde vi i slutändan ändå lägga till och ändra ganska mycket
men den initiella analysen var ändå en av de viktigaste momenten i hela projektet. \newline \newline 
Vi försökte dela upp arbetet så mycket som möjligt genom att arbeta parallellt med olika delar av projektet vilket fungerade bra och 
fungerade så att vi helt enkelt tog en post-it var från övre delen av klasshierarkin och kodade den. Det fungerade bra första
dagarna men när det inte var så många klasser kvar och vi började komma till implementeringen av klasserna så delades projektet
upp i två huvuddelar där vi då ansvarade för varsin. Det här var ingenting vi egentligen diskuterade utan det föll sig naturligt
i arbetsgången. Det ska sägas att vi satt bredvid varandra på campus och kodade under hela projektet  minimerade möjliga
kommunikationsmissar om den här typen av uppdelning av arbete vilket har varit bra.\newline \newline 
Den huvudsakliga uppdelningen av arbetet blev så att jag min partner skrev mer av den underliggande arkitekturen i spelet medan
jag skrev allt som syntes på skärmen, jag fick göra allt det roliga arbetet kan man säga. Med underliggande arkitektur
menar jag främst gameloopen som i vårt spel sköter i princip allt i spelet genom att dirigera melllan olika gamestates.
Så arbetet blev så att min partner skapade den och dess tillhörande gamestates medan jag då implementerade klasserna för
spelaren och de olika fiendetyperna. Eftersom vi jobbade parallelt så fanns det såklart ingen gameloop att köra vid testning
så under arbetets gång så hårdkodade jag i princip hela spelet i en separat mainfil för att kunna se om rörelser och kollisionshantering
fungerade som det skulle i spelet.\newline \newline 
Arbetetet blev rätt så tidspressat dels för att andra kurser har tagit tid men också för att vi började några dagar sent samt att vi som 
nämnt hade vissa besvär med en gruppmedlem i början av projektet. Tidspressen i sig har dock inte varit något större problem utan det har snarare varit ganska 
roligt att det har varit intensivt från början till slut, det beror antagligen mycket på att arbetet har gått bra de flesta dagarna och det hade
antagligen varit jobbigt annars. Vi hann iallafall klart till slut även om mycket om tilläggen som vi hade velat implementera inte har hunnits 
med.\newline
Överlag så har arbetsgången och arbetsuppdelningen varit bra under projektet, eftersom vi har suttit bredvid varandra hela tiden så
har vi kunnat hjälpas åt mycket och det har varit väldigt skönt att alltid ha nån att bolla idéer med. 
\newpage

\section{Ny programmeringskunskap}
Projektet har varit väldigt lärorik för min utveckling som programmerare då det har gjort att jag har fått mer kunskap och vana 
med att programmera i C++. Det har varit givande att använda SDL för att få en bättre förståelse för hur grafik
och programmering kan vara sammankopplat, jag vet inte hur andra liknande ramverk fungerar men SDL verkar som en bra grund
att arbeta med på den här nivån. Det känns även kul att veta att jag nu kan utveckla den här typen av program som faktiskt "syns"
mer än som text i ett terminalfönster och som kan reagera på flera typer av input från användaren. \newline  \newline 
Spelet som vi har skapat innehåller väldigt mycket rotation och vinklar då olika objekt flyger åt olika håll och snurrar och byter 
riktning hela tiden. Det har inneburit att jag har behövt läsa upp på gamla trigonometrikunskaper vilket har varit förvånansvärt intressant
och framförallt så känns det som att det kan vara bra kunskap att ha nära till hands för liknande projekt i framtiden eftersom
matematiken är densamma vid byte av programmeringsspråk.  \newline \newline 
Många koncept som inte var helt klara i C++ innan har även blivit lite tydligare under projektets gång, främst då hur man kan använda pekare och 
vektorer vilket känns som en väldigt viktig del av C++. \newline \newline
Den största och mest centrala lärdomen handlar däremot om objektorientering och klasser vilket jag känner att jag har lärt mig enormt mycket om
jämfört med det jag visste innan. Jag hade inte riktigt förstått poängen av objektorientering innan vi började implementera klasserna
som vi hade definierat i vårt spel och vi började få igång arv och pylomorfism och jag förstod precis hur användbart det kan vara. 
Det som jag tyckte var alldra bäst med det i vårt projekt var att vi använde en virtuell update funktion i alla objekt, det ledde
till lite frustation i början just för att vi ofta kände att vi ville kunna skicka med argument som behövdes för de olika objekten. 
Vi höll dock fast vid att ha en update som inte tog några argument och arbetade "runt" alla problem genom att använda setters, getters 
och medlemsvariabler och tur var väll det för det gjorde allting så grymt fint i gameloopen sen eftersom det enda vi behövde göra för 
att uppdatera spelet mellan iterationer var att gå igenom en vektor med pekare och köra update på objektet den pekade på. Hur smidigt det är
tycker jag är helt fantastiskt såhär på efterhand. Självklart så har kursen även gett mycket praktiskt kunskap om objektorientering
men för min del så är den största lärdomen från projektet en mer abstrakt förståelse av hur bra abstraktion är i programmering och hur
stor del objektorientering måste ha i större system med tanke på hur stor del det hade av vårt lilla spel. \newline \newline
En sak som kan vara värd att nämna är att vi inte använda eclipse i projektet överhuvudtaget, vilket såhär på efterhand är lite tråkigt
men samtidigt så har jag skrivit lite kod i visual studios förut så jag hade iallafall lite erfarenhet av att jobba med en IDE sedan tidigare.
Istället för Eclipse så använde vi oss av Emacs och terminalen.
   
\section{Kort sammanfattning}
Vi hade vissa problem med en gruppmedlem vilket jag tyvärr inte tycker att jag har lärt mig så mycket av mer än att jag har fått erfarenhet
av en situation där alla parter inte riktigt gör vad de ska. Arbetsgången gick dock bra ändå och det var ju egentligen inget
problem att vara två i en grupp eftersom de flesta grupperna var två. \newline \newline
Det jag tar med mig från projektet är främst en mycket större förståelse för objektorientering, en större kunskapsbas i C++ och SDL rörande både
koncept och syntax och även lite förnyade trigonometrikunskaper som kan vara nyttiga i framtiden. 


\end{document}
